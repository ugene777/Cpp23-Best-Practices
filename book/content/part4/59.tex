尽管编译器在不断改进,优化器也在努力处理这些类型的复杂性,但它们仍然很可能带来显著的编译时和运行时开销。

C++14 的 Lambda 表达式凭借其泛化的捕获功能,已经能够实现 std::bind 所能完成的所有事情。

\begin{myTip}{}
std::function 不支持 constexpr,这是另一个你可能需要避免在代码中使用它的原因。
\end{myTip}

\filename{图 115. 使用 'std::bind' 改变参数顺序}
\begin{cpp}
import std;

[[nodiscard]] constexpr double divide(double numerator, double denominator) {
  return numerator / denominator;
}

auto inverted_divide = std::bind(divide,
                                 std::placeholders::_2,
                                 std::placeholders::_1);
\end{cpp}

\filename{图 116. 使用 Lambda 改变参数顺序}
\begin{cpp}
import std;

[[nodiscard]] constexpr double divide(double numerator, double denominator) {
  return numerator / denominator;
}

auto inverted_divide = [](const auto numerator,
                          const auto denominator) {
  return divide(denominator, numerator)
}
\end{cpp}

\begin{myTip}{练习:比较各种可能性。}
在 Compiler Explorer 中尝试这些选项。它们的编译时间和生成的汇编代码是怎样的?

\filename{图 117. ‘std::function‘ 和 ‘std::bind‘}
\begin{cpp}
import std;

// std::function 不支持 constexpr,因此这里不能用 constexpr
[[nodiscard]] std::function<int (int)> bind_3(auto func)
{
  return std::bind(func, std::placeholders::_1, 3);
}

int main(int argc, const char *[])
{
  return bind_3(std::plus<>{})(argc);
}
\end{cpp}

\filename{图 118. 仅使用 ‘std::bind‘,额外加分项:函数 ‘bind\_3‘ 返回的类型是什么?}
\begin{cpp}
#include <functional>

// std::bind 支持 constexpr
[[nodiscard]] constexpr auto bind_3(auto func)
{
  return std::bind(func, std::placeholders::_1, 3);
}

int main(int argc, const char *[])
{
  return bind_3(std::plus<>{})(argc);
}
\end{cpp}

\filename{图 119. 仅使用 Lambdas,不使用任何 std 库包装器。}
\begin{cpp}
#include <functional>

[[nodiscard]] constexpr auto bind_3(auto func)
{
  return [func](const int value){ return func(value, 3); };
}

int main(int argc, const char *[])
{
  return bind_3(std::plus<>{})(argc);
}
\end{cpp}
\end{myTip}

\begin{myTip}{练习:查看 std::bind\_front, std::bind\_back}
\filename{图 120. C++23 的 bind\_back}
\begin{cpp}
#include <functional>

[[nodiscard]] constexpr auto bind_3(auto func)
{
  return std::bind_back(func, 3);
}

int main(int argc, const char *[])
{
  return bind_3(std::plus<>{})(argc);
}
\end{cpp}
\end{myTip}

\mySubsectionNoFile{59.1}{资源}

\begin{itemize}
\item
CppCon 2015: Stephan T. Lavavej “新特性介绍及正确用法\footnote{\url{https://youtu.be/zt7ThwVfap0}}

\item
C++ 周刊 第16集:“避免使用 std::bind”\footnote{\url{https://youtu.be/ZlHi8txU4aQ}}
\end{itemize}












