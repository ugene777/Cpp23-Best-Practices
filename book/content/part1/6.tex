本节旨在提醒我们,C++ 的所有方面都是可理解和推理的。它不是一个黑盒,也并非魔法。

如果你有疑问,通常可以设计一个实验来帮助自己找到答案。

我常用的一个工具是下面这个简单的类,它能在特殊成员函数调用时输出调试信息。

\filename{图 1. 理解对象生命周期的工具}

\begin{cpp}
import std;

struct S {
  S(){ std::println("S()"); }
  S(const S &){ std::println("S(const S &)"); }
  S(S &&){ std::println("S(S &&)"); }
  S &operator=(const S &){
    std::println("operator=(const S &)");
    return *this;
  }
  S &operator=(S &&){
    std::println("operator=(S &&)");
    return *this;
  }
  ~S() { std::println("~S()"); }
};
\end{cpp}

\begin{myTip}{练习:创建第一个 C++ 实验}
是否有一个一直困扰着你的 C++ 问题?你能设计一个实验来验证它吗?记住,Compiler Explorer 允许直接执行代码。
\end{myTip}

\begin{myTip}{练习:开始收集你的实验}
当你创建并完成了一个实验和测试,请务必将其保存下来。可以考虑使用 GitHub Gist 作为一种简单的方式来保存并与其他分享你的测试。
\end{myTip}

\mySubsectionNoFile{6.1}{资源}

\begin{itemize}
\item
Compiler Explorer 快速入门示例\footnote{\url{https://godbolt.org/z/3eGP56}}
\end{itemize}